\chapter{Introduction}

A representation is a group homomorphism whose image is a group of invertible matrices. At a glance, representations give us the ability to dial back the complexity of a mysterious group by viewing its elements as matrices. Thanks to the rigorous development of linear algebra, groups of matrices are well-understood structures. Representations allow us to unravel the mystery of any unknown group's structure and reveal a group’s fundamental properties as the result of linear algebra techniques. This is especially useful to us when a group is constructed to reflect the behavior of physical phenomena. In this thesis, the groups we will study create rigorous definitions for physical actions such as rotations, translations, relativistic transformations, and interactions of particles in quantum systems. In the search for representations of these groups, we utilize physical symmetries observed in the real world to make generalizable calculations. That is, we can use representations of groups that reflect simple physical actions to generate representations of the groups corresponding to more complicated physical actions. The primary objective in our study of representation theory is to seek a method to characterize properties of physical systems as the direct result of properties of the mathematical structures that underlie them.

Representation theory is already a well-documented and researched subject. In fact, the study of representation theory may be approached through the lens of many different fields in mathematics. To name a few, combinatorics, category theory, and abstract algebra all play a hand in the rigorous development of this subject. In this thesis, we sample many of these disciplines, finding connections between these seemingly disparate fields of study, to gain deeper insight into the structure of representations. For sake of time, this sampling of each discipline is nowhere near a complete examination of every facet of representation theory. Instead, the main work done in this thesis seeks to create a streamlined approach to studying representations while pointing out necessary developments from other branches of mathematics to make our examination rigorous. To this end, many of the theorems and concepts that different sources regard as accepted truth in their studies are explicitly proved here. It is the hope that readers should find the study of representation theory accessible even if they do not have expertise in the more niche fields of mathematics.
