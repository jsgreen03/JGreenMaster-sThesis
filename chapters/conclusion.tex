\chapter{Conclusion}
As we have seen, representation theory pulls insight from a diverse set of mathematical disciplines to create meaningful techniques to interpret even the most complicated groups. The representations we calculated in this thesis hold important physical significance. While we have only explored the representations of a handful of groups, representation theory can be used to characterize the behavior of many different physical systems. To briefly touch on a few, the Heisenberg group, the special unitary group, and the Galilean group can be analyzed in the by examining the physical symmetries that are inherent to their structure. Further, the study of representations of $C^*$-algebras, quantum groups, and Kac-Moody algebras are particularly relevant to theoretical physics in the realm of quantum field theory and exactly solvable models. Finally, generalizations of representation theory to categories give us surprisingly useful insight into the inner workings of physical systems. 

It is clear there is no shortage of avenues to explore the study of representation theory. While the further development of the mathematics and the physics required to explore more of this subject is nontrivial, the path illustrated gives rise to many opportunities to learn new methods to characterize the mathematical structure underpinning physical phenomena. Representations of these groups provide insight into the complex systems that govern our world.