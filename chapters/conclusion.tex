\chapter{Conclusion}
As we have seen, representation theory pulls insight from a diverse set of mathematical disciplines to create meaningful techniques to interpret even the most complicated groups. While we have only explored the representations of a handful of groups, representation theory can be used to characterize the behavior of many different physical systems. To briefly touch on a few, the study of representations of $C^*$-algebras and quantum groups are a mathematically rich endeavor. Further, the study of representations Kac-Moody algebras is particularly relevant to theoretical physics in the realm of conformal field theory and exactly solvable models. Finally, generalizations of representation theory to categories give us surprisingly useful insight into the inner workings of physical systems. 

The representations we calculated in this thesis hold important physical significance. As we saw, representations that are defined on vector spaces characterize the effect that physical actions have on spaces of particles and wave functions. While our study of groups representing physical actions was not exhaustive, we can see the general ideas that outline the process of studying the impact of groups on physical systems. There are still other groups that can be readily analyzed using these techniques, like the Heisenberg group, the special unitary group, and the Galilean group. We can utilize the inherent physical symmetries that lie at the heart of these structures to simplify our search for representations of these groups. 

It is clear there is no shortage of avenues to explore the study of representation theory. While the further development of the mathematics and the physics required to explore more of this subject is nontrivial, the path illustrated gives rise to many opportunities to learn new methods to characterize the mathematical structure underpinning physical phenomena. Representations of these groups provide insight into the complex systems that govern our world.