\documentclass[10pt]{ucthesis}

%\newif\ifpdf
%\ifx\pdfoutput\undefined
%    \pdffalse % we are not running PDFLaTeX
%\else
%\pdfoutput=1 % we are running PDFLaTeX
%\pdftrue \fi

\usepackage{url}
%\ifpdf

    \usepackage[pdftex]{graphicx}
    % Update title and author below...
    \usepackage[pdftex,plainpages=false,breaklinks=true,colorlinks=true,urlcolor=black,citecolor=black,%
                                       linkcolor=black,bookmarks=true,bookmarksopen=true,%
                                       bookmarksopenlevel=3,pdfstartview=FitV,
                                       pdfauthor={YOUR NAME},
                                       pdftitle={YOUR THESIS TITLE},
                                       pdfkeywords={thesis, masters, cal poly}
                                       ]{hyperref}
    %Options with pdfstartview are FitV, FitB and FitH
    \pdfcompresslevel=1

%\else
%    \usepackage{graphicx}
%\fi

\usepackage{booktabs} % To thicken table lines
\usepackage{amssymb}
\usepackage{amsmath}
\usepackage{mathtools}
\usepackage[letterpaper]{geometry}	
\usepackage[overload]{textcase}
\usepackage{amsthm}
\usepackage{algpseudocode}
\usepackage{array}
%\hypersetup{draft}
%\usepackage[draft]{hyperref}
%\usepackage{nohyperref}  % This makes hyperref commands do nothing without errors
%\usepackage{url}  % This makes \url work
%\usepackage[morefloats=125]{morefloats}
%\usepackage[hyphens]{url}
\usepackage{graphicx}
\usepackage{tabularx}
\usepackage{amssymb}
\usepackage{amsmath}
\usepackage[letterpaper]{geometry}
\usepackage[overload]{textcase}
\usepackage{color}
\usepackage[nonumberlist,toc]{glossaries}
\usepackage{wrapfig}
\usepackage{longtable}
\usepackage{morefloats}
\usepackage{float}
\usepackage{listings}
\usepackage{makecell}
\usepackage{appendix}
\usepackage[]{algorithm2e}
\usepackage{titlesec}
%\usepackage[breaklinks=true,hidelinks,pdfusetitle]{hyperref}
% \usepackage{cleveref}

\setcounter{secnumdepth}{3}
\setcounter{tocdepth}{3}

% Added to avoid windows and orphans
\usepackage[all]{nowidow}
% Added to fix spacing between footnote entries
\usepackage{setspace}


\newcommand{\Z}{\mathbb{Z}}
\newcommand{\R}{\mathbb{R}}
\newcommand{\N}{\mathbb{N}}
\newtheorem{definition}{Definition}[chapter]
\newtheorem{theorem}[definition]{Theorem}
\newtheorem{example}[definition]{Example}
\newtheorem{algo}[definition]{Algorithm}
\newtheorem{lemma}[definition]{Lemma}
\newtheorem{assumption}[definition]{Assumption}


%\bibliographystyle{abbrv}

\setlength{\parindent}{0.25in} \setlength{\parskip}{6pt}

\geometry{verbose,nohead,tmargin=1in,bmargin=1in,lmargin=1.5in,rmargin=1.5in}

\setcounter{tocdepth}{2}


% Different font in captions (single-spaced, bold) ------------
\newcommand{\captionfonts}{\small\bf\ssp}

\makeatletter  % Allow the use of @ in command names
\long\def\@makecaption#1#2{%
  \vskip\abovecaptionskip
  \sbox\@tempboxa{{\captionfonts #1: #2}}%
  \ifdim \wd\@tempboxa >\hsize
    {\captionfonts #1: #2\par}
  \else
    \hbox to\hsize{\hfil\box\@tempboxa\hfil}%
  \fi
  \vskip\belowcaptionskip}
\makeatother   % Cancel the effect of \makeatletter
% ---------------------------------------

   
\titleformat{\chapter}[hang]
{\normalfont%
    % \huge% %change this size to your needs for the first line
    \bfseries}{\chaptertitlename\ \thechapter}{10pt}{%
    % \huge %change this size to your needs for the second line
    }
    
\titleformat{\section}[hang]
{\normalfont%
    % \huge% %change this size to your needs for the first line
    \bfseries}{ \thesection}{10pt}{%
    % \huge %change this size to your needs for the second line
    }

% \renewcommand*{\chapterheadendvskip}{%
%   \vspace{0.725\baselineskip plus 0.115\baselineskip minus 0.192\baselineskip}%
% }

\titleformat{\chapter}[display]
        {\normalfont\normalsize\centering%\bfseries
        }
        {\ifthenelse{\equal{\thechapter}{A}}{APPENDICES\\[4.3ex]}{}\chaptertitlename\ \thechapter}
        {0pt}{\normalsize\uppercase}
\titlespacing*{\chapter}{0pt}{-10pt}{4.3ex plus .2ex}


\titleformat*{\section}{\normalsize%\bfseries
}
\titleformat*{\subsection}{\small%\bfseries
}
\titleformat*{\subsubsection}{\small%\bfseries
}
\titleformat*{\paragraph}{\small%\bfseries
}
\titleformat*{\subparagraph}{\small%\bfseries}
}

% \hypersetup{nolinks=true}
\begin{document}
\hypersetup{nolinks=true}


% Declarations for Front Matter

% Update fields below!
\title{Representation Theory in Braid Groups}
\author{Jaxon Green}
\degreeyear{2024}
\degreesemester{June}
\degree{Master of Science}
\chair{Professor Ben Richert\\  Professor of Mathematics} 
\othermembers{Professor Sean Gasiorek}
%\othermemberA{OTHER MEMBER HERE \\ & Professor of Mathematics}
%\othermemberB{OTHER MEMBER HERE\\ & Professor of Mathematics} 
\prevdegrees{None}
\numberofmembers{1}
\field{Mathematics} 
\campus{San Luis Obispo}
%\copyrightyears{seven}



\maketitle

\begin{frontmatter}

\copyrightpage


\begin{abstract}

Write an abstract here.

\vspace*{-10pt}
\end{abstract}

\begin{acknowledgements}

Any acknowledgements?

\end{acknowledgements}

\tableofcontents


\listoftables

\listoffigures

% Add CHAPTER into table of contents.

%\addtocontents{toc}{%
   %\noindent Representation Theory\\
   %\noindent Representations of Groups in Physics
%}

\end{frontmatter}

\pagestyle{plain}




\renewcommand{\baselinestretch}{1.66}


% \chapter{CHAPTER}
% ------------- Main chapters here --------------------
\chapter{Sample Chapter 1}
\label{intro}

 \section{First Section of Introduction}
 \label{intro1}
 This is an equation:
 \begin{equation}
 c^2=a^2+b^2.
 \end{equation}










\chapter{Sample Chapter 2}

\section{MY FIRST SECTION}
There are lots of great resources on the internet to help you learn \LaTeX.  
Perhaps start with examples like the ones at 
\begin{verbatim}http://en.wikibooks.org/wiki/LaTeX/Sample_LaTeX_documents.\end{verbatim} 

It is important to cite references.
%\cite{Blum}  \cite{Gill} \cite{Ped2}
\cite{Blum, Gill, Ped2}

Organize the paper into sections and subsections.  

\subsection{Interesting subsection title}

You get the idea.  Hey, this one has some displayed math, 
\[
   \frac{2}{x} = \sin(\epsilon), 
\]
not that it makes any sense whatsoever.  And here is how you 
do a numbered equation, 
\begin{equation}
   \int_{0}^{y^2} f(x) \, dx = \sqrt{z+y}.  
\end{equation}
Don't forget to punctuate your equations as part of the sentence.  
You can do inline math, too, as in $f(x) = \lim_{n\to \infty} n f(x)/n$, which is trivial. 

\subsection{Another interesting subsection title}

Okay, not really.  



\begin{figure}[htbp] %  figure placement: here, top, bottom, or page
   \centering
   \includegraphics[width=4in]{meme.jpg} 
   \caption{This is a sample figure.  It is not interesting. Note that 
   figures will ``float'' to wherever \LaTeX \ wants to put them.  }
   \label{fig:example}
\end{figure}


\chapter{Representation Theory}

\section{Introduction to Representations}

Over the course of this chapter, we will develop the theory and utility of representations. At a glance, representations give us the ability to dial back the complexity of a mysterious group by viewing its elements as matrices. Thanks to the rigorous development and study of linear algebra, groups of matrices are well-understood structures. Representations allow us to unravel the mystery of an unknown group structure and reveal a group's fundamental properties as results of linear algebra techniques. 

\begin{definition}
	A \textbf{representation} of degree n is a group homomorphism that maps a group into $GL_n(\mathbb{C})$
\end{definition}

To illustrate to idea of a representation, we will consider the group of all roots of unity ($G$). We can construct multiple homomorphisms from $G$ to showcase different kinds of representations. 

\noindent Note: we will be treating $GL_1(\mathbb{C})$ as $\mathbb{C}$ since they are isomorphic.

\begin{example}
	(Trivial Representation)\\\\
	\renewcommand{\arraystretch}{0.7}
	Let   \begin{tabular}{l}$\phi:G\rightarrow GL_1(\mathbb{C})$\\
		$\hspace{6mm}g\mapsto 1$
		\end{tabular} \\\\
	This map is the trivial homomorphism from $G$ to $GL_1(\mathbb{C})$ and therefore it easily satisfies the requirement of a degree $1$ representation of $G$. We say that $\phi$ is the \textbf{trivial representation} of $G$.
\end{example}
	

\begin{example}
	(Nontrivial Degree 1 Representation) \\
	By construction of $G$, if $g\in G$, then $g = e^{\frac{2\pi im }{n}}$ where $m,n\in \mathbb{Z}$	
	\renewcommand{\arraystretch}{0.7}\\\\
	Let   \begin{tabular}{l}$\phi:G\rightarrow GL_1(\mathbb{C})$\\
		$\hspace{6mm}g\mapsto g$
		\end{tabular} \\\\
	where we view $G$ as a multiplicative subgroup of $\mathbb{C}$. This observation trivializes the argument that $\phi$ is a homomorphism. Therefore, $\phi$ is a degree $1$ representation of $G$.		
\end{example}

\begin{example}
	(Degree 2 Representation) \\\\
	\renewcommand{\arraystretch}{1}
	Let   \begin{tabular}{l}$\phi:G\rightarrow GL_2(\mathbb{C})$\\
		$\hspace{0mm}e^{2\pi i\frac{m}{n}}\mapsto \begin{bmatrix}
							\cos(\frac{2\pi m}{n}) & \sin(\frac{2\pi m}{n}) \\
							-\sin(\frac{2\pi m}{n}) & \cos(\frac{2\pi m}{n})
						      \end{bmatrix}$
		\end{tabular} \\\\
	To show this map is a homomorphism, we will take two elements of $G$, say $e^{2\pi i\frac{x}{y}}$ and $e^{2\pi i\frac{a }{b}}$ and track the image of their product under $\phi$. \\
	\begin{equation}
		\begin{aligned}
			\phi(e^{2\pi i\frac{x}{y}} * e^{2\pi i\frac{a}{b}} ) &= \phi(e^{2\pi i(\frac{x}{y}+\frac{a}{b})})\\ 
												    &= \begin{bmatrix}
														\cos(2\pi(\frac{x}{y}+\frac{a}{b})) & \sin(2\pi(\frac{x}{y}+\frac{a}{b})) \\
														-\sin(2\pi(\frac{x}{y}+\frac{a}{b})) & \cos(2\pi(\frac{x}{y}+\frac{a}{b}))
													  \end{bmatrix}\\
												    &= \begin{bmatrix}
														\cos(2\pi\frac{x}{y})\cos(2\pi\frac{a}{b}) - \sin(2\pi\frac{x}{y})\sin(2\pi\frac{a}{b})   &\sin(2\pi\frac{x}{y})\cos(2\pi\frac{a}{b}) + \cos(2\pi\frac{x}{y})\sin(2\pi\frac{a}{b})\\
														-\sin(2\pi\frac{x}{y})\cos(2\pi\frac{a}{b}) - \cos(2\pi\frac{x}{y})\sin(2\pi\frac{a}{b}) & \cos(2\pi\frac{x}{y})\cos(2\pi\frac{a}{b}) - \sin(2\pi\frac{x}{y})\sin(2\pi\frac{a}{b})
													  \end{bmatrix}\\ 
												    &= \begin{bmatrix}
														\cos(2\pi\frac{x}{y}) & \sin(2\pi\frac{x}{y}) \\
														-\sin(2\pi\frac{x}{y}) & \cos(2\pi\frac{x}{y})
												          \end{bmatrix}
											  		  \begin{bmatrix}
														\cos(2\pi\frac{a}{b}) & \sin(2\pi\frac{a}{b}) \\
														-\sin(2\pi\frac{a}{b}) & \cos(2\pi\frac{a}{b})
													  \end{bmatrix} \\
		                                                                                    &= \phi(e^{2\pi i\frac{x}{y}})*\phi(e^{2\pi i\frac{a}{b}})
		\end{aligned}
	\end{equation}
	Since, $\phi$ has been shown to be a homomorphism, we can conclude that $\phi$ is also a degree $2$ representation of $G$.
\end{example}

As illustrated in the previous example, the process of confirming a map is a representation is relatively standard. However, there does not seem to be any intuitive way to come up with a new representation. The rest of this chapter will be devoted the process of comparing and characterizing every representation of a given group.

%Making new ones, comparing two, characterize all? 

\begin{definition}
	Two representations, $\phi$ and $\psi$, are said to be \textbf{equivalent representations} if there exists some invertible group homomorphism, $\mu$, such that $$\phi = \mu \psi \mu^{-1}$$
\end{definition}






% ------------- End main chapters ----------------------



\clearpage

%%%%%%%%%%%%%%%%%%%%%%%%%%%%%%%%%%%%%
%% the following two commands set up the bibliography and references 
%% automatically throughout the document.  The file my_special_bibliography.bib 
%% is one you create with all the info about all your references.  The alpha.bst file 
%% is included in your LaTeX distribution, but you can modify it if you want.  (But 
%% you don't want.)  
%\bibliography{my_special_ibliography}
%\bibliographystyle{alpha}
%%%%%%%%%%%%%%%%%%%%%%%%%%%%%%%%%%%%%

\begin{thebibliography}{9}
\bibliographystyle{IEEEtran}

\bibitem{Blum}
A.~F. Blumberg and G.~L. Mellor.
\newblock A description of a three-dimensional coastal circulation model.
\newblock In N.~Heaps, editor, {\em Three Dimensional Coastal Ocean Models},
  pages 1--16. Amer. Geophys. Union, 1987.

\bibitem{Gill}
A.~Gill.
\newblock {\em Atmosphere - {O}cean {D}ynamics}.
\newblock Academic Press, 1982.

\bibitem{Chob}
P.~F. Choboter, R.~M. Samelson, and J.~S. Allen.
\newblock A {N}ew {S}olution of a {N}onlinear {M}odel of {U}pwelling.
\newblock {\em J. Phys Oceanogr.}, 35:532--544, 2005.

\bibitem{Lentz}
S.~J. Lentz and D.~C. Chapman.
\newblock The importance of non-linear cross-shelf momentum flux during
  wind-driven coastal upwelling.
\newblock {\em J. Phys. Oceanogr.}, 34:2444--2457, 2004.

\bibitem{Ped2}
J.~Pedlosky.
\newblock A {N}onlinear {M}odel of the {O}nset of {U}pwelling.
\newblock {\em J. Phys Oceanogr.}, 8:178--187, 1978.

\end{thebibliography}


% Indents Appendix in Table of Contents
\makeatletter
\addtocontents{toc}{\let\protect\l@chapter\protect\l@section}
\makeatother

% Hack to make Appendices to appear in Table of Contents
\addtocontents{toc}{%
   \noindent APPENDICES
}

\begin{appendices}
\chapter{Appendix A Title} \label{Appendix A}

Blah blah

\end{appendices}


%\addcontentsline{toc}{chapter}{Bibliography}

\end{document}